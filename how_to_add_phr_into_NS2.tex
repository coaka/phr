\documentclass{article}        

\definecolor{lightgray}{gray}{0.9}
\usepackage{listings}
\usepackage{color}
\usepackage{xcolor}
\definecolor{dkgreen}{rgb}{0,0.6,0}
\definecolor{gray}{rgb}{0.5,0.5,0.5}
%
\definecolor{mauve}{rgb}{0.58,0,0.82}
\begin{document}               
\lstset{frame=tb,
  language=C++,
  aboveskip=3mm,
  belowskip=3mm,
  showstringspaces=false,
  columns=flexible,
  frame=lines,
  numbers=left,
  basicstyle={\small\ttfamily},
  numbers=left,
  numberstyle=\tiny\color{gray},
  keywordstyle=\color{blue},
  commentstyle=\color{dkgreen},
  stringstyle=\color{mauve},
  breaklines=true,
  breakatwhitespace=true,
  tabsize=3
}


\title{Integrate PHR source code into NS2}

\emph{PHR} directory contains three files, i.e. \emph{phr.cc}, \emph{phr.h}, \emp	 and \emph{phr\_packet.h}.
In order to integrate and compile \emph{PHR} protocol in \emph{NS2}, the following steps should fulfilled.

\begin{itemize}
\item It should install a fresh copy of \emph{NS2.35}.
\item Download and copy \emph{PHR} directory into \emph{~/ns-allinone-2.35/ns-2.35/}.
\item Add \emph{case PT\_PHR} to \emph{~/ns-allinone-2.35/ns-2.35/queue/priqueue.cc} from line $94$.
\item \emph{PHR} packet header need to be defined, \emph{~/ns-allinone-2.35/ns-2.35/common/packet.h} file
should modified accordingly by adding \emph{\#define HDR\_PHR(p)      (hdr\_phr::access(p))} after line $62$.

\item Modifying same file, PT\_NTYPE should change to $74$, and for \emph{PHR} protocol PT\_PHR = $73$. 
If you have already installed another routing protocol. Just make sure PT\_NTYPE is last, and protocol number is ordered 
sequentially. Code in~\ref{pkt-h} shows the changes to \emph{packet.h}.

\lstset{language=C++}
\begin{lstlisting}[caption= Packet header file changes \label{pkt-h} ,float]
static const packet_t PT_PHR = 73;
static packet_t PT_NTYPE = 74; // This MUST be the LAST one
\end{lstlisting}

\item Add the \emph{type == PT\_PHR} as shown in~\ref{pkt-h} at line $280$ of the same \emph{packet.h}.
\begin{lstlisting}[caption= Make packet has high priority \label{pkt-h} ,float]
type == PT_PHR ||  
type == PT_MDART)
\end{lstlisting}
Then add \emph{name\_[PT\_PHR]="PHR"} in line $420$.
  
\item In order to provide a trace functionality into the simulation, it should enable \emph{NS2} to trace
all the events in the simulation, to do that, \emph{~/ns-allinone-2.35/ns-2.35/trace/cmu-trace.h \& cmu-trace.cc}
 files need to be modified.
\item First, define drop reasons by adding lines in~\ref{drp} into \emph{cmu-trace.h} at line$85$. 
\begin{lstlisting}[caption= Define drop reasons \label{drp} ,float]
#define DROP_PHR_PH_CLOSER            "CLOSER"//PH is closer to D
#define DROP_PHR_DKNOW                "DKNOW"// know flag set and don't know about D.
\end{lstlisting}

\item Define trace function in \emph{cmu-trace.h} at line $165$ as shown in~\ref{trace-fun}. 
\begin{lstlisting}[caption= Define trace function \label{trace-fun} ]
void    format_phr(Packet *p, int offset);
\end{lstlisting}

\item The implementation of the trace function should be added in \emph{cmu-trace.cc} at line $1182$ as shown in~\ref{trace-fun-imp}.
\begin{lstlisting}[caption= Main body of PHR trace function. \label{trace-fun-imp} ]
#include <phr/phr_packet.h>   //PHR protocol
// main body of the trace function.\
void
CMUTrace::format_phr(Packet * p, int offset)
	{struct hdr_phr *phr = HDR_PHR(p);
	struct hdr_phr_bc *bc = HDR_PHR_BC(p);
	switch (phr->pkt_type) {
	case PHR_BC:
		if (pt_->tagged())
			{sprintf(pt_->buffer() + offset,
				"-PHR:t %x -PHR:h %d -PHR:b %d -PHR:s %d "
				"-PHR:ts %f "
				"-PHR:c PHR ",
				bc->bc_type,
				bc->bc_hop_count,
				bc->bc_bcast_id,
				bc->bc_src,
				bc->bc_timestamp);
		} else if (newtrace_)
		 	{sprintf(pt_->buffer() + offset,
				"-P phr -Pt 0x%x -Ph %d -Pb %d -Ps %d -Pts %f -Pc PHR ",
				bc->bc_type,
				bc->bc_hop_count,
				bc->bc_bcast_id,
				bc->bc_src,
				bc->bc_timestamp);
		} else {sprintf(pt_->buffer() + offset,
				"[0x%x %d %d [%d] [%f]] (PHR)",
				bc->bc_type,
				bc->bc_hop_count,
				bc->bc_bcast_id,
				bc->bc_src,
				bc->bc_timestamp);
		}
		break;
	default:
#ifdef WIN32
		fprintf(stderr,
			"CMUTrace::format_phr: invalid PHR packet typen");
#else
		fprintf(stderr,
			"%s: invalid PHR packet typen", __FUNCTION__);
#endif
		abort();
	}
}
\end{lstlisting}

\item After changing C++ files, TCL files also need to be changed to create \emph{PHR} routing agent to be 
 used in TCL file. This is done by modifying \emph{~/ns-allinone-2.35/ns-2.35/tcl/lib/ns-packet.tcl}.
\item Add \emph{PHR} at line $172$
\item Set routing agent by modifying \emph{~/ns-allinone-2.35/ns-2.35/tcl/lib/ns-lib.tcl} at line $639$ as
shown in~\ref{set-phr}.

\lstset{language=TCL}
\begin{lstlisting}[caption= Set PHR agent \label{set-phr} ,float]
PHR {
  set ragent [$self create-phr-agent $node]
}
\item At line $870$ code in \~\ref{tcl} should be added.
\begin{lstlisting}[caption= Create PHR agent \label{tcl}, float]
Simulator instproc create-phr-agent {node}
	 {set ragent [new Agent/PHR [$node node-addr]]
   $self at 0.0 "$ragent start"
   $node set ragent_ $ragent
   return $ragent
}
\end{lstlisting}
 
\item Set port numbers of \emph{PHR} agent (sport is the source port, dport is destination port)
 by adding code in~\ref{sdport} to \emph{~/ns-allinone-2.35/ns-2.35/tcl/lib/ns-agent.tcl} at line $201$.
\begin{lstlisting}[caption=Set ports of PHR agent \label{sdport} ,float]
Agent/PHR instproc init args
	 {$self next $args
}
Agent/PHR set sport_   0
Agent/PHR set dport_   0
\end{lstlisting}


\item Modify \emph{~/ns-allinone-2.35/ns-2.35/tcl/lib/ns-mobilenode.tcl} by adding code in~\ref{mn} at line $204$.
\begin{lstlisting}[caption=Set ports of PHR agent \label{mn} ,float]
# Special processing for PHR
set phronly [string first "PHR" [$agent info class]]
if {$phronly!= -1} 
	{$agent if-queue [$self set ifq_(0)]   ;# ifq between LL and MAC
}
\end{lstlisting}


\item Modify \emph{~/ns-allinone-2.35/ns-2.35/Makefile} by adding \emph{phr\/phr.o} after \emph{puma\/puma.o } line
 to the list of object files for \emph{NS2}.
 
 \end{itemize}
Now, \emph{NS2} should be ready to be recompiled. To do so, run make command in 
\emph{~/ns-allinone-2.35/ns-2.35/}. When the compilation is done, \emph{NS2} ready to be tested with \emph{PHR}.



\end{document}



